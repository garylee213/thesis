\chapter{Conclusion}
For stationary atoms, an analytical calculation of radiated fields is feasible in simple cases such as a single atom, two atoms, and a non-cooperative Gaussian sample. We derived explicit expressions of total radiation power in these cases. The discrepancy between a continuous medium and a sample of discrete non-interacting atoms touches some fundamental problems in the traditional optics which is essentially a mean-field theory. More details are discussed in other works.

We carried out simulations for stationary atoms with dipole-dipole interactions. A direct comparison of the total radiated power with independent atoms has revealed the cooperative effects.

In the second part of this paper, we incorporated atomic motion, including collisions, in our classical-electrodynamics simulations of light propagation in a dense gas sample. The introduction of atomic motion not only reproduces the same results as we achieved on stationary atoms, such as collective Lamb shift, but also demonstrates new spectral features that result from cooperative effects on the optical response of the gas.

Starting from Eq.~\eq{DIPOLEEQ}, our calculations and simulations on moving atoms confirms that for a gas confined in a certain container like a circular disk that we adopted, if each atom is supposed to be polarized solely by the incident light but electromagnetically independent of any other atoms, a shorter mean free path would narrow the absorption spectrum. This is consistent with the conventional theory of Dick narrowing.

However, a confirmative signature of Dick narrowing, i.e., a sharp Lorentzian peak over a wide Gaussian base, would manifest only when dipole-dipole interactions are included in the simulations. Even more interesting, the cooperative response in this system reverses the dependence of the line width on the gas density. An extra broadening is observed based on a Dick-narrowed spectrum. ~\cite{Sci.325.1510}.
