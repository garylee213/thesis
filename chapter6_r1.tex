\chapter{Concluding Remarks}
For stationary atoms, an analytical calculation of radiated fields is feasible in simple cases such as a single atom, two atoms, and a non-cooperative Gaussian sample. We have derived explicit expressions for the total radiated power in these cases. The difference between a continuous medium and a sample of discrete non-interacting atoms touches on some fundamental problems in traditional optics, which is essentially a mean-field theory.

We have carried out simulations for stationary atoms with the dipole-dipole interactions taken into account. A comparison of the total radiated power with independent atoms has revealed the presence of cooperative effects. We also note that the shift of the atomic resonance line in the simulations is not compatible with the conventional wisdom about local-field corrections.

We incorporated atomic motion, including various collisions, in our classical-electrodynamics simulations of light propagation in a dense gas sample. The simulations with atomic motion not only qualitatively reproduced the same results as we found for stationary atoms with a simple model of inhomogeneous broadening, such as the collective Lamb shift, but also demonstrated new spectral features that result from the interplay between Dicke narrowing and cooperative effects.

At this point it is becoming obvious that we have only seen the tip of the iceberg. References~\cite{PhysRevLett.112.113603} and~\cite{Javanainen:16} increasingly insistently promote the idea that the usual electrodynamics of polarizable media may fail qualitatively in the kind of cold dense gases brought to us by laser cooling and evaporative cooling. Conversely, it is becoming clear that the quintessential cooperative Lamb shift in a slab geometry can be explained as a near-trivial consequence of classical optics. Let us bring in another thought directly related to the present thesis: We have seen line broadening with increasing atom density in our simulations, evidently as a result of cooperative atom-atom interactions. Now, conventional wisdom says that resonance lines {\em do\/} broaden with atom density; it is called collision broadening. Could it be that collision broadening is actually a cooperative effect due to dipole-dipole interactions?

We conclude with the following theses: The molecular basis of the electrodynamics of material samples is probably not as well understood as we have thought until now. Numerical simulations on massive computer clusters are a new theoretical method to address this problem area, and the progress in experimental techniques with cold atoms will eventually enable decisive experiments. Classical electrodynamics, a field whose fundamentals were seemingly set in stone well over a century ago, is due for new developments.
